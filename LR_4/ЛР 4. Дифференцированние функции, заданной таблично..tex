\documentclass[11pt]{article}

    \author{ПИН-21 Чендемеров Алексей}

    \usepackage[breakable]{tcolorbox}
    \usepackage{parskip} % Stop auto-indenting (to mimic markdown behaviour)
    
    \usepackage[utf8]{inputenc}
    \usepackage[russian,english]{babel}
    \selectlanguage{russian}

    \usepackage{iftex}
    \ifPDFTeX
    	\usepackage[T1]{fontenc}
    	\usepackage{mathpazo}
    \else
    	\usepackage{fontspec}
    \fi
    \setmainfont{Linux Libertine}
    \newfontfamily\russianfont[Script=Cyrillic]{Linux Libertine}

    % Basic figure setup, for now with no caption control since it's done
    % automatically by Pandoc (which extracts ![](path) syntax from Markdown).
    \usepackage{graphicx}
    % Maintain compatibility with old templates. Remove in nbconvert 6.0
    \let\Oldincludegraphics\includegraphics
    % Ensure that by default, figures have no caption (until we provide a
    % proper Figure object with a Caption API and a way to capture that
    % in the conversion process - todo).
    \usepackage{caption}
    \DeclareCaptionFormat{nocaption}{}
    \captionsetup{format=nocaption,aboveskip=0pt,belowskip=0pt}

    \usepackage{float}
    \floatplacement{figure}{H} % forces figures to be placed at the correct location
    \usepackage{xcolor} % Allow colors to be defined
    \usepackage{enumerate} % Needed for markdown enumerations to work
    \usepackage{geometry} % Used to adjust the document margins
    \usepackage{amsmath} % Equations
    \usepackage{amssymb} % Equations
    \usepackage{textcomp} % defines textquotesingle
    % Hack from http://tex.stackexchange.com/a/47451/13684:
    \AtBeginDocument{%
        \def\PYZsq{\textquotesingle}% Upright quotes in Pygmentized code
    }
    \usepackage{upquote} % Upright quotes for verbatim code
    \usepackage{eurosym} % defines \euro
    \usepackage[mathletters]{ucs} % Extended unicode (utf-8) support
    \usepackage{fancyvrb} % verbatim replacement that allows latex
    \usepackage{grffile} % extends the file name processing of package graphics 
                         % to support a larger range
    \makeatletter % fix for old versions of grffile with XeLaTeX
    \@ifpackagelater{grffile}{2019/11/01}
    {
      % Do nothing on new versions
    }
    {
      \def\Gread@@xetex#1{%
        \IfFileExists{"\Gin@base".bb}%
        {\Gread@eps{\Gin@base.bb}}%
        {\Gread@@xetex@aux#1}%
      }
    }
    \makeatother
    \usepackage[Export]{adjustbox} % Used to constrain images to a maximum size
    \adjustboxset{max size={0.9\linewidth}{0.9\paperheight}}

    % The hyperref package gives us a pdf with properly built
    % internal navigation ('pdf bookmarks' for the table of contents,
    % internal cross-reference links, web links for URLs, etc.)
    \usepackage{hyperref}
    % The default LaTeX title has an obnoxious amount of whitespace. By default,
    % titling removes some of it. It also provides customization options.
    \usepackage{titling}
    \usepackage{longtable} % longtable support required by pandoc >1.10
    \usepackage{booktabs}  % table support for pandoc > 1.12.2
    \usepackage[inline]{enumitem} % IRkernel/repr support (it uses the enumerate* environment)
    \usepackage[normalem]{ulem} % ulem is needed to support strikethroughs (\sout)
                                % normalem makes italics be italics, not underlines
    \usepackage{mathrsfs}
    

    
    % Colors for the hyperref package
    \definecolor{urlcolor}{rgb}{0,.145,.698}
    \definecolor{linkcolor}{rgb}{.71,0.21,0.01}
    \definecolor{citecolor}{rgb}{.12,.54,.11}

    % ANSI colors
    \definecolor{ansi-black}{HTML}{3E424D}
    \definecolor{ansi-black-intense}{HTML}{282C36}
    \definecolor{ansi-red}{HTML}{E75C58}
    \definecolor{ansi-red-intense}{HTML}{B22B31}
    \definecolor{ansi-green}{HTML}{00A250}
    \definecolor{ansi-green-intense}{HTML}{007427}
    \definecolor{ansi-yellow}{HTML}{DDB62B}
    \definecolor{ansi-yellow-intense}{HTML}{B27D12}
    \definecolor{ansi-blue}{HTML}{208FFB}
    \definecolor{ansi-blue-intense}{HTML}{0065CA}
    \definecolor{ansi-magenta}{HTML}{D160C4}
    \definecolor{ansi-magenta-intense}{HTML}{A03196}
    \definecolor{ansi-cyan}{HTML}{60C6C8}
    \definecolor{ansi-cyan-intense}{HTML}{258F8F}
    \definecolor{ansi-white}{HTML}{C5C1B4}
    \definecolor{ansi-white-intense}{HTML}{A1A6B2}
    \definecolor{ansi-default-inverse-fg}{HTML}{FFFFFF}
    \definecolor{ansi-default-inverse-bg}{HTML}{000000}

    % common color for the border for error outputs.
    \definecolor{outerrorbackground}{HTML}{FFDFDF}

    % commands and environments needed by pandoc snippets
    % extracted from the output of `pandoc -s`
    \providecommand{\tightlist}{%
      \setlength{\itemsep}{0pt}\setlength{\parskip}{0pt}}
    \DefineVerbatimEnvironment{Highlighting}{Verbatim}{commandchars=\\\{\}}
    % Add ',fontsize=\small' for more characters per line
    \newenvironment{Shaded}{}{}
    \newcommand{\KeywordTok}[1]{\textcolor[rgb]{0.00,0.44,0.13}{\textbf{{#1}}}}
    \newcommand{\DataTypeTok}[1]{\textcolor[rgb]{0.56,0.13,0.00}{{#1}}}
    \newcommand{\DecValTok}[1]{\textcolor[rgb]{0.25,0.63,0.44}{{#1}}}
    \newcommand{\BaseNTok}[1]{\textcolor[rgb]{0.25,0.63,0.44}{{#1}}}
    \newcommand{\FloatTok}[1]{\textcolor[rgb]{0.25,0.63,0.44}{{#1}}}
    \newcommand{\CharTok}[1]{\textcolor[rgb]{0.25,0.44,0.63}{{#1}}}
    \newcommand{\StringTok}[1]{\textcolor[rgb]{0.25,0.44,0.63}{{#1}}}
    \newcommand{\CommentTok}[1]{\textcolor[rgb]{0.38,0.63,0.69}{\textit{{#1}}}}
    \newcommand{\OtherTok}[1]{\textcolor[rgb]{0.00,0.44,0.13}{{#1}}}
    \newcommand{\AlertTok}[1]{\textcolor[rgb]{1.00,0.00,0.00}{\textbf{{#1}}}}
    \newcommand{\FunctionTok}[1]{\textcolor[rgb]{0.02,0.16,0.49}{{#1}}}
    \newcommand{\RegionMarkerTok}[1]{{#1}}
    \newcommand{\ErrorTok}[1]{\textcolor[rgb]{1.00,0.00,0.00}{\textbf{{#1}}}}
    \newcommand{\NormalTok}[1]{{#1}}
    
    % Additional commands for more recent versions of Pandoc
    \newcommand{\ConstantTok}[1]{\textcolor[rgb]{0.53,0.00,0.00}{{#1}}}
    \newcommand{\SpecialCharTok}[1]{\textcolor[rgb]{0.25,0.44,0.63}{{#1}}}
    \newcommand{\VerbatimStringTok}[1]{\textcolor[rgb]{0.25,0.44,0.63}{{#1}}}
    \newcommand{\SpecialStringTok}[1]{\textcolor[rgb]{0.73,0.40,0.53}{{#1}}}
    \newcommand{\ImportTok}[1]{{#1}}
    \newcommand{\DocumentationTok}[1]{\textcolor[rgb]{0.73,0.13,0.13}{\textit{{#1}}}}
    \newcommand{\AnnotationTok}[1]{\textcolor[rgb]{0.38,0.63,0.69}{\textbf{\textit{{#1}}}}}
    \newcommand{\CommentVarTok}[1]{\textcolor[rgb]{0.38,0.63,0.69}{\textbf{\textit{{#1}}}}}
    \newcommand{\VariableTok}[1]{\textcolor[rgb]{0.10,0.09,0.49}{{#1}}}
    \newcommand{\ControlFlowTok}[1]{\textcolor[rgb]{0.00,0.44,0.13}{\textbf{{#1}}}}
    \newcommand{\OperatorTok}[1]{\textcolor[rgb]{0.40,0.40,0.40}{{#1}}}
    \newcommand{\BuiltInTok}[1]{{#1}}
    \newcommand{\ExtensionTok}[1]{{#1}}
    \newcommand{\PreprocessorTok}[1]{\textcolor[rgb]{0.74,0.48,0.00}{{#1}}}
    \newcommand{\AttributeTok}[1]{\textcolor[rgb]{0.49,0.56,0.16}{{#1}}}
    \newcommand{\InformationTok}[1]{\textcolor[rgb]{0.38,0.63,0.69}{\textbf{\textit{{#1}}}}}
    \newcommand{\WarningTok}[1]{\textcolor[rgb]{0.38,0.63,0.69}{\textbf{\textit{{#1}}}}}
    
    
    % Define a nice break command that doesn't care if a line doesn't already
    % exist.
    \def\br{\hspace*{\fill} \\* }
    % Math Jax compatibility definitions
    \def\gt{>}
    \def\lt{<}
    \let\Oldtex\TeX
    \let\Oldlatex\LaTeX
    \renewcommand{\TeX}{\textrm{\Oldtex}}
    \renewcommand{\LaTeX}{\textrm{\Oldlatex}}
    % Document parameters
    % Document title
    \title{ЛР 4. Дифференцированние функции, заданной таблично.}
    
    
    
    
    
% Pygments definitions
\makeatletter
\def\PY@reset{\let\PY@it=\relax \let\PY@bf=\relax%
    \let\PY@ul=\relax \let\PY@tc=\relax%
    \let\PY@bc=\relax \let\PY@ff=\relax}
\def\PY@tok#1{\csname PY@tok@#1\endcsname}
\def\PY@toks#1+{\ifx\relax#1\empty\else%
    \PY@tok{#1}\expandafter\PY@toks\fi}
\def\PY@do#1{\PY@bc{\PY@tc{\PY@ul{%
    \PY@it{\PY@bf{\PY@ff{#1}}}}}}}
\def\PY#1#2{\PY@reset\PY@toks#1+\relax+\PY@do{#2}}

\@namedef{PY@tok@w}{\def\PY@tc##1{\textcolor[rgb]{0.73,0.73,0.73}{##1}}}
\@namedef{PY@tok@c}{\let\PY@it=\textit\def\PY@tc##1{\textcolor[rgb]{0.25,0.50,0.50}{##1}}}
\@namedef{PY@tok@cp}{\def\PY@tc##1{\textcolor[rgb]{0.74,0.48,0.00}{##1}}}
\@namedef{PY@tok@k}{\let\PY@bf=\textbf\def\PY@tc##1{\textcolor[rgb]{0.00,0.50,0.00}{##1}}}
\@namedef{PY@tok@kp}{\def\PY@tc##1{\textcolor[rgb]{0.00,0.50,0.00}{##1}}}
\@namedef{PY@tok@kt}{\def\PY@tc##1{\textcolor[rgb]{0.69,0.00,0.25}{##1}}}
\@namedef{PY@tok@o}{\def\PY@tc##1{\textcolor[rgb]{0.40,0.40,0.40}{##1}}}
\@namedef{PY@tok@ow}{\let\PY@bf=\textbf\def\PY@tc##1{\textcolor[rgb]{0.67,0.13,1.00}{##1}}}
\@namedef{PY@tok@nb}{\def\PY@tc##1{\textcolor[rgb]{0.00,0.50,0.00}{##1}}}
\@namedef{PY@tok@nf}{\def\PY@tc##1{\textcolor[rgb]{0.00,0.00,1.00}{##1}}}
\@namedef{PY@tok@nc}{\let\PY@bf=\textbf\def\PY@tc##1{\textcolor[rgb]{0.00,0.00,1.00}{##1}}}
\@namedef{PY@tok@nn}{\let\PY@bf=\textbf\def\PY@tc##1{\textcolor[rgb]{0.00,0.00,1.00}{##1}}}
\@namedef{PY@tok@ne}{\let\PY@bf=\textbf\def\PY@tc##1{\textcolor[rgb]{0.82,0.25,0.23}{##1}}}
\@namedef{PY@tok@nv}{\def\PY@tc##1{\textcolor[rgb]{0.10,0.09,0.49}{##1}}}
\@namedef{PY@tok@no}{\def\PY@tc##1{\textcolor[rgb]{0.53,0.00,0.00}{##1}}}
\@namedef{PY@tok@nl}{\def\PY@tc##1{\textcolor[rgb]{0.63,0.63,0.00}{##1}}}
\@namedef{PY@tok@ni}{\let\PY@bf=\textbf\def\PY@tc##1{\textcolor[rgb]{0.60,0.60,0.60}{##1}}}
\@namedef{PY@tok@na}{\def\PY@tc##1{\textcolor[rgb]{0.49,0.56,0.16}{##1}}}
\@namedef{PY@tok@nt}{\let\PY@bf=\textbf\def\PY@tc##1{\textcolor[rgb]{0.00,0.50,0.00}{##1}}}
\@namedef{PY@tok@nd}{\def\PY@tc##1{\textcolor[rgb]{0.67,0.13,1.00}{##1}}}
\@namedef{PY@tok@s}{\def\PY@tc##1{\textcolor[rgb]{0.73,0.13,0.13}{##1}}}
\@namedef{PY@tok@sd}{\let\PY@it=\textit\def\PY@tc##1{\textcolor[rgb]{0.73,0.13,0.13}{##1}}}
\@namedef{PY@tok@si}{\let\PY@bf=\textbf\def\PY@tc##1{\textcolor[rgb]{0.73,0.40,0.53}{##1}}}
\@namedef{PY@tok@se}{\let\PY@bf=\textbf\def\PY@tc##1{\textcolor[rgb]{0.73,0.40,0.13}{##1}}}
\@namedef{PY@tok@sr}{\def\PY@tc##1{\textcolor[rgb]{0.73,0.40,0.53}{##1}}}
\@namedef{PY@tok@ss}{\def\PY@tc##1{\textcolor[rgb]{0.10,0.09,0.49}{##1}}}
\@namedef{PY@tok@sx}{\def\PY@tc##1{\textcolor[rgb]{0.00,0.50,0.00}{##1}}}
\@namedef{PY@tok@m}{\def\PY@tc##1{\textcolor[rgb]{0.40,0.40,0.40}{##1}}}
\@namedef{PY@tok@gh}{\let\PY@bf=\textbf\def\PY@tc##1{\textcolor[rgb]{0.00,0.00,0.50}{##1}}}
\@namedef{PY@tok@gu}{\let\PY@bf=\textbf\def\PY@tc##1{\textcolor[rgb]{0.50,0.00,0.50}{##1}}}
\@namedef{PY@tok@gd}{\def\PY@tc##1{\textcolor[rgb]{0.63,0.00,0.00}{##1}}}
\@namedef{PY@tok@gi}{\def\PY@tc##1{\textcolor[rgb]{0.00,0.63,0.00}{##1}}}
\@namedef{PY@tok@gr}{\def\PY@tc##1{\textcolor[rgb]{1.00,0.00,0.00}{##1}}}
\@namedef{PY@tok@ge}{\let\PY@it=\textit}
\@namedef{PY@tok@gs}{\let\PY@bf=\textbf}
\@namedef{PY@tok@gp}{\let\PY@bf=\textbf\def\PY@tc##1{\textcolor[rgb]{0.00,0.00,0.50}{##1}}}
\@namedef{PY@tok@go}{\def\PY@tc##1{\textcolor[rgb]{0.53,0.53,0.53}{##1}}}
\@namedef{PY@tok@gt}{\def\PY@tc##1{\textcolor[rgb]{0.00,0.27,0.87}{##1}}}
\@namedef{PY@tok@err}{\def\PY@bc##1{{\setlength{\fboxsep}{\string -\fboxrule}\fcolorbox[rgb]{1.00,0.00,0.00}{1,1,1}{\strut ##1}}}}
\@namedef{PY@tok@kc}{\let\PY@bf=\textbf\def\PY@tc##1{\textcolor[rgb]{0.00,0.50,0.00}{##1}}}
\@namedef{PY@tok@kd}{\let\PY@bf=\textbf\def\PY@tc##1{\textcolor[rgb]{0.00,0.50,0.00}{##1}}}
\@namedef{PY@tok@kn}{\let\PY@bf=\textbf\def\PY@tc##1{\textcolor[rgb]{0.00,0.50,0.00}{##1}}}
\@namedef{PY@tok@kr}{\let\PY@bf=\textbf\def\PY@tc##1{\textcolor[rgb]{0.00,0.50,0.00}{##1}}}
\@namedef{PY@tok@bp}{\def\PY@tc##1{\textcolor[rgb]{0.00,0.50,0.00}{##1}}}
\@namedef{PY@tok@fm}{\def\PY@tc##1{\textcolor[rgb]{0.00,0.00,1.00}{##1}}}
\@namedef{PY@tok@vc}{\def\PY@tc##1{\textcolor[rgb]{0.10,0.09,0.49}{##1}}}
\@namedef{PY@tok@vg}{\def\PY@tc##1{\textcolor[rgb]{0.10,0.09,0.49}{##1}}}
\@namedef{PY@tok@vi}{\def\PY@tc##1{\textcolor[rgb]{0.10,0.09,0.49}{##1}}}
\@namedef{PY@tok@vm}{\def\PY@tc##1{\textcolor[rgb]{0.10,0.09,0.49}{##1}}}
\@namedef{PY@tok@sa}{\def\PY@tc##1{\textcolor[rgb]{0.73,0.13,0.13}{##1}}}
\@namedef{PY@tok@sb}{\def\PY@tc##1{\textcolor[rgb]{0.73,0.13,0.13}{##1}}}
\@namedef{PY@tok@sc}{\def\PY@tc##1{\textcolor[rgb]{0.73,0.13,0.13}{##1}}}
\@namedef{PY@tok@dl}{\def\PY@tc##1{\textcolor[rgb]{0.73,0.13,0.13}{##1}}}
\@namedef{PY@tok@s2}{\def\PY@tc##1{\textcolor[rgb]{0.73,0.13,0.13}{##1}}}
\@namedef{PY@tok@sh}{\def\PY@tc##1{\textcolor[rgb]{0.73,0.13,0.13}{##1}}}
\@namedef{PY@tok@s1}{\def\PY@tc##1{\textcolor[rgb]{0.73,0.13,0.13}{##1}}}
\@namedef{PY@tok@mb}{\def\PY@tc##1{\textcolor[rgb]{0.40,0.40,0.40}{##1}}}
\@namedef{PY@tok@mf}{\def\PY@tc##1{\textcolor[rgb]{0.40,0.40,0.40}{##1}}}
\@namedef{PY@tok@mh}{\def\PY@tc##1{\textcolor[rgb]{0.40,0.40,0.40}{##1}}}
\@namedef{PY@tok@mi}{\def\PY@tc##1{\textcolor[rgb]{0.40,0.40,0.40}{##1}}}
\@namedef{PY@tok@il}{\def\PY@tc##1{\textcolor[rgb]{0.40,0.40,0.40}{##1}}}
\@namedef{PY@tok@mo}{\def\PY@tc##1{\textcolor[rgb]{0.40,0.40,0.40}{##1}}}
\@namedef{PY@tok@ch}{\let\PY@it=\textit\def\PY@tc##1{\textcolor[rgb]{0.25,0.50,0.50}{##1}}}
\@namedef{PY@tok@cm}{\let\PY@it=\textit\def\PY@tc##1{\textcolor[rgb]{0.25,0.50,0.50}{##1}}}
\@namedef{PY@tok@cpf}{\let\PY@it=\textit\def\PY@tc##1{\textcolor[rgb]{0.25,0.50,0.50}{##1}}}
\@namedef{PY@tok@c1}{\let\PY@it=\textit\def\PY@tc##1{\textcolor[rgb]{0.25,0.50,0.50}{##1}}}
\@namedef{PY@tok@cs}{\let\PY@it=\textit\def\PY@tc##1{\textcolor[rgb]{0.25,0.50,0.50}{##1}}}

\def\PYZbs{\char`\\}
\def\PYZus{\char`\_}
\def\PYZob{\char`\{}
\def\PYZcb{\char`\}}
\def\PYZca{\char`\^}
\def\PYZam{\char`\&}
\def\PYZlt{\char`\<}
\def\PYZgt{\char`\>}
\def\PYZsh{\char`\#}
\def\PYZpc{\char`\%}
\def\PYZdl{\char`\$}
\def\PYZhy{\char`\-}
\def\PYZsq{\char`\'}
\def\PYZdq{\char`\"}
\def\PYZti{\char`\~}
% for compatibility with earlier versions
\def\PYZat{@}
\def\PYZlb{[}
\def\PYZrb{]}
\makeatother


    % For linebreaks inside Verbatim environment from package fancyvrb. 
    \makeatletter
        \newbox\Wrappedcontinuationbox 
        \newbox\Wrappedvisiblespacebox 
        \newcommand*\Wrappedvisiblespace {\textcolor{red}{\textvisiblespace}} 
        \newcommand*\Wrappedcontinuationsymbol {\textcolor{red}{\llap{\tiny$\m@th\hookrightarrow$}}} 
        \newcommand*\Wrappedcontinuationindent {3ex } 
        \newcommand*\Wrappedafterbreak {\kern\Wrappedcontinuationindent\copy\Wrappedcontinuationbox} 
        % Take advantage of the already applied Pygments mark-up to insert 
        % potential linebreaks for TeX processing. 
        %        {, <, #, %, $, ' and ": go to next line. 
        %        _, }, ^, &, >, - and ~: stay at end of broken line. 
        % Use of \textquotesingle for straight quote. 
        \newcommand*\Wrappedbreaksatspecials {% 
            \def\PYGZus{\discretionary{\char`\_}{\Wrappedafterbreak}{\char`\_}}% 
            \def\PYGZob{\discretionary{}{\Wrappedafterbreak\char`\{}{\char`\{}}% 
            \def\PYGZcb{\discretionary{\char`\}}{\Wrappedafterbreak}{\char`\}}}% 
            \def\PYGZca{\discretionary{\char`\^}{\Wrappedafterbreak}{\char`\^}}% 
            \def\PYGZam{\discretionary{\char`\&}{\Wrappedafterbreak}{\char`\&}}% 
            \def\PYGZlt{\discretionary{}{\Wrappedafterbreak\char`\<}{\char`\<}}% 
            \def\PYGZgt{\discretionary{\char`\>}{\Wrappedafterbreak}{\char`\>}}% 
            \def\PYGZsh{\discretionary{}{\Wrappedafterbreak\char`\#}{\char`\#}}% 
            \def\PYGZpc{\discretionary{}{\Wrappedafterbreak\char`\%}{\char`\%}}% 
            \def\PYGZdl{\discretionary{}{\Wrappedafterbreak\char`\$}{\char`\$}}% 
            \def\PYGZhy{\discretionary{\char`\-}{\Wrappedafterbreak}{\char`\-}}% 
            \def\PYGZsq{\discretionary{}{\Wrappedafterbreak\textquotesingle}{\textquotesingle}}% 
            \def\PYGZdq{\discretionary{}{\Wrappedafterbreak\char`\"}{\char`\"}}% 
            \def\PYGZti{\discretionary{\char`\~}{\Wrappedafterbreak}{\char`\~}}% 
        } 
        % Some characters . , ; ? ! / are not pygmentized. 
        % This macro makes them "active" and they will insert potential linebreaks 
        \newcommand*\Wrappedbreaksatpunct {% 
            \lccode`\~`\.\lowercase{\def~}{\discretionary{\hbox{\char`\.}}{\Wrappedafterbreak}{\hbox{\char`\.}}}% 
            \lccode`\~`\,\lowercase{\def~}{\discretionary{\hbox{\char`\,}}{\Wrappedafterbreak}{\hbox{\char`\,}}}% 
            \lccode`\~`\;\lowercase{\def~}{\discretionary{\hbox{\char`\;}}{\Wrappedafterbreak}{\hbox{\char`\;}}}% 
            \lccode`\~`\:\lowercase{\def~}{\discretionary{\hbox{\char`\:}}{\Wrappedafterbreak}{\hbox{\char`\:}}}% 
            \lccode`\~`\?\lowercase{\def~}{\discretionary{\hbox{\char`\?}}{\Wrappedafterbreak}{\hbox{\char`\?}}}% 
            \lccode`\~`\!\lowercase{\def~}{\discretionary{\hbox{\char`\!}}{\Wrappedafterbreak}{\hbox{\char`\!}}}% 
            \lccode`\~`\/\lowercase{\def~}{\discretionary{\hbox{\char`\/}}{\Wrappedafterbreak}{\hbox{\char`\/}}}% 
            \catcode`\.\active
            \catcode`\,\active 
            \catcode`\;\active
            \catcode`\:\active
            \catcode`\?\active
            \catcode`\!\active
            \catcode`\/\active 
            \lccode`\~`\~ 	
        }
    \makeatother

    \let\OriginalVerbatim=\Verbatim
    \makeatletter
    \renewcommand{\Verbatim}[1][1]{%
        %\parskip\z@skip
        \sbox\Wrappedcontinuationbox {\Wrappedcontinuationsymbol}%
        \sbox\Wrappedvisiblespacebox {\FV@SetupFont\Wrappedvisiblespace}%
        \def\FancyVerbFormatLine ##1{\hsize\linewidth
            \vtop{\raggedright\hyphenpenalty\z@\exhyphenpenalty\z@
                \doublehyphendemerits\z@\finalhyphendemerits\z@
                \strut ##1\strut}%
        }%
        % If the linebreak is at a space, the latter will be displayed as visible
        % space at end of first line, and a continuation symbol starts next line.
        % Stretch/shrink are however usually zero for typewriter font.
        \def\FV@Space {%
            \nobreak\hskip\z@ plus\fontdimen3\font minus\fontdimen4\font
            \discretionary{\copy\Wrappedvisiblespacebox}{\Wrappedafterbreak}
            {\kern\fontdimen2\font}%
        }%
        
        % Allow breaks at special characters using \PYG... macros.
        \Wrappedbreaksatspecials
        % Breaks at punctuation characters . , ; ? ! and / need catcode=\active 	
        \OriginalVerbatim[#1,codes*=\Wrappedbreaksatpunct]%
    }
    \makeatother

    % Exact colors from NB
    \definecolor{incolor}{HTML}{303F9F}
    \definecolor{outcolor}{HTML}{D84315}
    \definecolor{cellborder}{HTML}{CFCFCF}
    \definecolor{cellbackground}{HTML}{F7F7F7}
    
    % prompt
    \makeatletter
    \newcommand{\boxspacing}{\kern\kvtcb@left@rule\kern\kvtcb@boxsep}
    \makeatother
    \newcommand{\prompt}[4]{
        {\ttfamily\llap{{\color{#2}[#3]:\hspace{3pt}#4}}\vspace{-\baselineskip}}
    }
    

    
    % Prevent overflowing lines due to hard-to-break entities
    \sloppy 
    % Setup hyperref package
    \hypersetup{
      breaklinks=true,  % so long urls are correctly broken across lines
      colorlinks=true,
      urlcolor=urlcolor,
      linkcolor=linkcolor,
      citecolor=citecolor,
      }
    % Slightly bigger margins than the latex defaults
    
    \geometry{verbose,tmargin=1in,bmargin=1in,lmargin=1in,rmargin=1in}
    
    

\begin{document}
    
    \maketitle
    
    

    
    \hypertarget{ux43bux440-4.-ux434ux438ux444ux444ux435ux440ux435ux43dux446ux438ux440ux43eux432ux430ux43dux43dux438ux435-ux444ux443ux43dux43aux446ux438ux438-ux437ux430ux434ux430ux43dux43dux43eux439-ux442ux430ux431ux43bux438ux447ux43dux43e.}{%
\section{ЛР 4. Дифференцированние функции, заданной
таблично.}\label{ux43bux440-4.-ux434ux438ux444ux444ux435ux440ux435ux43dux446ux438ux440ux43eux432ux430ux43dux43dux438ux435-ux444ux443ux43dux43aux446ux438ux438-ux437ux430ux434ux430ux43dux43dux43eux439-ux442ux430ux431ux43bux438ux447ux43dux43e.}}

    \begin{tcolorbox}[breakable, size=fbox, boxrule=1pt, pad at break*=1mm,colback=cellbackground, colframe=cellborder]
\prompt{In}{incolor}{1}{\boxspacing}
\begin{Verbatim}[commandchars=\\\{\}]
\PY{k+kn}{import} \PY{n+nn}{numpy} \PY{k}{as} \PY{n+nn}{np}
\PY{k+kn}{import} \PY{n+nn}{sympy} \PY{k}{as} \PY{n+nn}{sp}
\PY{k+kn}{import} \PY{n+nn}{pandas} \PY{k}{as} \PY{n+nn}{pd}
\PY{k+kn}{import} \PY{n+nn}{math}
\PY{k+kn}{import} \PY{n+nn}{matplotlib}\PY{n+nn}{.}\PY{n+nn}{pyplot} \PY{k}{as} \PY{n+nn}{plt}
\PY{o}{\PYZpc{}}\PY{k}{matplotlib} inline
\PY{n}{plt}\PY{o}{.}\PY{n}{rcParams}\PY{p}{[}\PY{l+s+s2}{\PYZdq{}}\PY{l+s+s2}{figure.figsize}\PY{l+s+s2}{\PYZdq{}}\PY{p}{]} \PY{o}{=} \PY{p}{(}\PY{l+m+mi}{15}\PY{p}{,}\PY{l+m+mi}{10}\PY{p}{)}
\PY{n}{plt}\PY{o}{.}\PY{n}{rcParams}\PY{p}{[}\PY{l+s+s1}{\PYZsq{}}\PY{l+s+s1}{lines.linewidth}\PY{l+s+s1}{\PYZsq{}}\PY{p}{]} \PY{o}{=} \PY{l+m+mi}{2}
\end{Verbatim}
\end{tcolorbox}

    \hypertarget{ux437ux430ux434ux430ux43dux438ux435-1}{%
\subsection{Задание 1}\label{ux437ux430ux434ux430ux43dux438ux435-1}}

Выберите некоторую функцию (например, \(sin(x)\), \(cos(x)\),
\(exp(x)\), \(sh(x)\), \(ch(x)\), \(ln(x)\), \ldots{} ) и некоторую
точку \(x\) из области определения функции. Найдите значение производной
функции в выбранной точке (используя любую формулу численного
дифференцирования) с точностью \(10^{−3}\) , \(10^{−6}\) . Пользоваться
точным значением производной в качестве эталона запрещено.

    \begin{tcolorbox}[breakable, size=fbox, boxrule=1pt, pad at break*=1mm,colback=cellbackground, colframe=cellborder]
\prompt{In}{incolor}{2}{\boxspacing}
\begin{Verbatim}[commandchars=\\\{\}]
\PY{n}{f} \PY{o}{=} \PY{k}{lambda} \PY{n}{x}\PY{p}{:} \PY{n}{np}\PY{o}{.}\PY{n}{exp}\PY{p}{(}\PY{n}{x}\PY{p}{)}

\PY{k}{def} \PY{n+nf}{Rh2}\PY{p}{(}\PY{n}{F3}\PY{p}{,} \PY{n}{x0}\PY{p}{,} \PY{n}{h}\PY{p}{)}\PY{p}{:}
    \PY{n}{x} \PY{o}{=} \PY{n}{np}\PY{o}{.}\PY{n}{linspace}\PY{p}{(}\PY{n}{x0}\PY{o}{\PYZhy{}}\PY{n}{h}\PY{p}{,} \PY{n}{x0}\PY{o}{+}\PY{n}{h}\PY{p}{)}
    \PY{n}{y} \PY{o}{=} \PY{n+nb}{abs}\PY{p}{(}\PY{n}{F3}\PY{p}{(}\PY{n}{x}\PY{p}{)} \PY{o}{*} \PY{n}{h}\PY{o}{*}\PY{o}{*}\PY{l+m+mi}{2} \PY{o}{/} \PY{l+m+mi}{6}\PY{p}{)}
    \PY{k}{return} \PY{n+nb}{max}\PY{p}{(}\PY{n}{y}\PY{p}{)}

\PY{k}{def} \PY{n+nf}{derivative}\PY{p}{(}\PY{n}{f}\PY{p}{,} \PY{n}{F3}\PY{p}{,} \PY{n}{x0}\PY{p}{,} \PY{n}{eps}\PY{p}{)}\PY{p}{:}
    \PY{n}{h} \PY{o}{=} \PY{l+m+mi}{1}
    \PY{k}{while} \PY{n}{Rh2}\PY{p}{(}\PY{n}{F3}\PY{p}{,} \PY{n}{x0}\PY{p}{,} \PY{n}{h}\PY{p}{)} \PY{o}{\PYZgt{}}\PY{o}{=} \PY{n}{eps}\PY{p}{:}
        \PY{n}{h} \PY{o}{=} \PY{n}{h} \PY{o}{/} \PY{l+m+mi}{2}
    \PY{k}{return} \PY{p}{(}\PY{n}{f}\PY{p}{(}\PY{n}{x0}\PY{o}{+}\PY{n}{h}\PY{p}{)} \PY{o}{\PYZhy{}} \PY{n}{f}\PY{p}{(}\PY{n}{x0}\PY{o}{\PYZhy{}}\PY{n}{h}\PY{p}{)}\PY{p}{)} \PY{o}{/} \PY{p}{(}\PY{l+m+mi}{2}\PY{o}{*}\PY{n}{h}\PY{p}{)}

\PY{n+nb}{print}\PY{p}{(}\PY{l+s+s1}{\PYZsq{}}\PY{l+s+s1}{diff(exp(x)) = exp(x)}\PY{l+s+s1}{\PYZsq{}}\PY{p}{)}
\PY{n}{F3} \PY{o}{=} \PY{n}{np}\PY{o}{.}\PY{n}{exp}
\PY{n}{x0} \PY{o}{=} \PY{l+m+mi}{2}
\PY{n}{eps1} \PY{o}{=} \PY{l+m+mf}{1e\PYZhy{}3}
\PY{n}{eps2} \PY{o}{=} \PY{l+m+mf}{1e\PYZhy{}6}
\PY{n+nb}{print}\PY{p}{(}\PY{l+s+sa}{f}\PY{l+s+s1}{\PYZsq{}}\PY{l+s+s1}{Exp(x) at x0 = }\PY{l+s+si}{\PYZob{}}\PY{n}{x0}\PY{l+s+si}{\PYZcb{}}\PY{l+s+s1}{:}\PY{l+s+s1}{\PYZsq{}}\PY{p}{,} \PY{n}{np}\PY{o}{.}\PY{n}{exp}\PY{p}{(}\PY{n}{x0}\PY{p}{)}\PY{p}{)}
\PY{n+nb}{print}\PY{p}{(}\PY{l+s+sa}{f}\PY{l+s+s1}{\PYZsq{}}\PY{l+s+s1}{Diff of exp(x) at x0 = }\PY{l+s+si}{\PYZob{}}\PY{n}{x0}\PY{l+s+si}{\PYZcb{}}\PY{l+s+s1}{ with eps = }\PY{l+s+si}{\PYZob{}}\PY{n}{eps1}\PY{l+s+si}{\PYZcb{}}\PY{l+s+s1}{:}\PY{l+s+s1}{\PYZsq{}}\PY{p}{,} \PY{n}{derivative}\PY{p}{(}\PY{n}{f}\PY{p}{,} \PY{n}{F3}\PY{p}{,} \PY{n}{x0}\PY{p}{,} \PY{n}{eps1}\PY{p}{)}\PY{p}{)}
\PY{n+nb}{print}\PY{p}{(}\PY{l+s+sa}{f}\PY{l+s+s1}{\PYZsq{}}\PY{l+s+s1}{Diff of exp(x) at x0 = }\PY{l+s+si}{\PYZob{}}\PY{n}{x0}\PY{l+s+si}{\PYZcb{}}\PY{l+s+s1}{ with eps = }\PY{l+s+si}{\PYZob{}}\PY{n}{eps2}\PY{l+s+si}{\PYZcb{}}\PY{l+s+s1}{:}\PY{l+s+s1}{\PYZsq{}}\PY{p}{,} \PY{n}{derivative}\PY{p}{(}\PY{n}{f}\PY{p}{,} \PY{n}{F3}\PY{p}{,} \PY{n}{x0}\PY{p}{,} \PY{n}{eps2}\PY{p}{)}\PY{p}{)}
\end{Verbatim}
\end{tcolorbox}

    \begin{Verbatim}[commandchars=\\\{\}]
diff(exp(x)) = exp(x)
Exp(x) at x0 = 2: 7.38905609893065
Diff of exp(x) at x0 = 2 with eps = 0.001: 7.389356764063223
Diff of exp(x) at x0 = 2 with eps = 1e-06: 7.3890563925451715
    \end{Verbatim}

    Мы знаем, что \(e'(x) = e(x)\). Будем использовать это для самопроверки.
Для численного дифференцирования выбрал формулу центральной разности
\(f'(x) = \frac{f(x+h) - f(x-h)}{2h} + O(h^{2})\), где
\(O(h^{2}) = \max{\frac{h^{2}f'''(\eta)}{6}}, \eta \in [x-h, x+h]\) -
погрешность вычисления.

    \hypertarget{ux437ux430ux434ux430ux43dux438ux435-2}{%
\section{Задание 2}\label{ux437ux430ux434ux430ux43dux438ux435-2}}

Выберите некоторую функцию (например, \(sin(x)\), \(cos(x)\),
\(exp(x)\), \(sh(x)\), \(ch(x)\), \(ln(x)\), \ldots{} ) и некоторую
точку \(x\) из области определения функции. Сравните погрешности у
формул с разными порядками погрешностей (например,
\(y'(x) \approx \frac{y(x+h)−y(x)}{h}\) и
\(y'(x) \approx \frac{y(x+h)−y(x-h)}{2h}\)) для последовательности
убывающих шагов (например, \(h = \frac{1}{2}, \frac{1}{4}, \frac{1}{8}\)
). С какими скоростями убывают погрешности для каждой формулы? Дайте
теоретическую оценку и подтвердите ответ экспериментом.

    \begin{tcolorbox}[breakable, size=fbox, boxrule=1pt, pad at break*=1mm,colback=cellbackground, colframe=cellborder]
\prompt{In}{incolor}{3}{\boxspacing}
\begin{Verbatim}[commandchars=\\\{\}]
\PY{n}{f} \PY{o}{=} \PY{k}{lambda} \PY{n}{x}\PY{p}{:} \PY{n}{np}\PY{o}{.}\PY{n}{exp}\PY{p}{(}\PY{n}{x}\PY{p}{)}

\PY{k}{def} \PY{n+nf}{Rh}\PY{p}{(}\PY{n}{F2}\PY{p}{,} \PY{n}{x0}\PY{p}{,} \PY{n}{h}\PY{p}{)}\PY{p}{:}
    \PY{n}{x} \PY{o}{=} \PY{n}{np}\PY{o}{.}\PY{n}{linspace}\PY{p}{(}\PY{n}{x0}\PY{p}{,} \PY{n}{x0}\PY{o}{+}\PY{n}{h}\PY{p}{)}
    \PY{n}{y} \PY{o}{=} \PY{n+nb}{abs}\PY{p}{(}\PY{n}{F2}\PY{p}{(}\PY{n}{x}\PY{p}{)} \PY{o}{*} \PY{n}{h} \PY{o}{/} \PY{l+m+mi}{2}\PY{p}{)}
    \PY{k}{return} \PY{n+nb}{max}\PY{p}{(}\PY{n}{y}\PY{p}{)}

\PY{k}{def} \PY{n+nf}{Rh2}\PY{p}{(}\PY{n}{F3}\PY{p}{,} \PY{n}{x0}\PY{p}{,} \PY{n}{h}\PY{p}{)}\PY{p}{:}
    \PY{n}{x} \PY{o}{=} \PY{n}{np}\PY{o}{.}\PY{n}{linspace}\PY{p}{(}\PY{n}{x0}\PY{o}{\PYZhy{}}\PY{n}{h}\PY{p}{,} \PY{n}{x0}\PY{o}{+}\PY{n}{h}\PY{p}{)}
    \PY{n}{y} \PY{o}{=} \PY{n+nb}{abs}\PY{p}{(}\PY{n}{F3}\PY{p}{(}\PY{n}{x}\PY{p}{)} \PY{o}{*} \PY{n}{h}\PY{o}{*}\PY{o}{*}\PY{l+m+mi}{2} \PY{o}{/} \PY{l+m+mi}{6}\PY{p}{)}
    \PY{k}{return} \PY{n+nb}{max}\PY{p}{(}\PY{n}{y}\PY{p}{)}

\PY{k}{def} \PY{n+nf}{der\PYZus{}formula1}\PY{p}{(}\PY{n}{f}\PY{p}{,} \PY{n}{x0}\PY{p}{,} \PY{n}{h}\PY{p}{)}\PY{p}{:}
    \PY{k}{return} \PY{p}{(}\PY{n}{f}\PY{p}{(}\PY{n}{x0}\PY{o}{+}\PY{n}{h}\PY{p}{)} \PY{o}{\PYZhy{}} \PY{n}{f}\PY{p}{(}\PY{n}{x0}\PY{p}{)}\PY{p}{)} \PY{o}{/} \PY{n}{h}

\PY{k}{def} \PY{n+nf}{der\PYZus{}formula2}\PY{p}{(}\PY{n}{f}\PY{p}{,} \PY{n}{x0}\PY{p}{,} \PY{n}{h}\PY{p}{)}\PY{p}{:}
    \PY{k}{return} \PY{p}{(}\PY{n}{f}\PY{p}{(}\PY{n}{x0}\PY{o}{+}\PY{n}{h}\PY{p}{)} \PY{o}{\PYZhy{}} \PY{n}{f}\PY{p}{(}\PY{n}{x0}\PY{o}{\PYZhy{}}\PY{n}{h}\PY{p}{)}\PY{p}{)} \PY{o}{/} \PY{p}{(}\PY{l+m+mi}{2}\PY{o}{*}\PY{n}{h}\PY{p}{)}

\PY{n}{x0} \PY{o}{=} \PY{l+m+mi}{2}
\PY{n}{F2} \PY{o}{=} \PY{n}{np}\PY{o}{.}\PY{n}{exp}
\PY{n}{F3} \PY{o}{=} \PY{n}{np}\PY{o}{.}\PY{n}{exp}
\PY{n}{h} \PY{o}{=} \PY{n}{np}\PY{o}{.}\PY{n}{array}\PY{p}{(}\PY{p}{[}\PY{l+m+mi}{1}\PY{p}{,} \PY{l+m+mi}{1}\PY{o}{/}\PY{l+m+mi}{2}\PY{p}{,} \PY{l+m+mi}{1}\PY{o}{/}\PY{l+m+mi}{4}\PY{p}{,} \PY{l+m+mi}{1}\PY{o}{/}\PY{l+m+mi}{8}\PY{p}{,} \PY{l+m+mi}{1}\PY{o}{/}\PY{l+m+mi}{16}\PY{p}{]}\PY{p}{)}
\PY{n}{Oh} \PY{o}{=} \PY{n}{np}\PY{o}{.}\PY{n}{array}\PY{p}{(}\PY{p}{[}\PY{n}{Rh}\PY{p}{(}\PY{n}{F2}\PY{p}{,} \PY{n}{x0}\PY{p}{,} \PY{n}{c}\PY{p}{)} \PY{k}{for} \PY{n}{c} \PY{o+ow}{in} \PY{n}{h}\PY{p}{]}\PY{p}{)}
\PY{n}{Oh2} \PY{o}{=} \PY{n}{np}\PY{o}{.}\PY{n}{array}\PY{p}{(}\PY{p}{[}\PY{n}{Rh2}\PY{p}{(}\PY{n}{F3}\PY{p}{,} \PY{n}{x0}\PY{p}{,} \PY{n}{c}\PY{p}{)} \PY{k}{for} \PY{n}{c} \PY{o+ow}{in} \PY{n}{h}\PY{p}{]}\PY{p}{)}
\PY{n}{ideal} \PY{o}{=} \PY{n}{np}\PY{o}{.}\PY{n}{array}\PY{p}{(}\PY{p}{[}\PY{n}{f}\PY{p}{(}\PY{n}{x0}\PY{p}{)}\PY{p}{]} \PY{o}{*} \PY{n+nb}{len}\PY{p}{(}\PY{n}{h}\PY{p}{)}\PY{p}{)}
\PY{n}{f1\PYZus{}values} \PY{o}{=} \PY{n}{np}\PY{o}{.}\PY{n}{array}\PY{p}{(}\PY{p}{[}\PY{n}{der\PYZus{}formula1}\PY{p}{(}\PY{n}{f}\PY{p}{,} \PY{n}{x0}\PY{p}{,} \PY{n}{c}\PY{p}{)} \PY{k}{for} \PY{n}{c} \PY{o+ow}{in}  \PY{n}{h}\PY{p}{]}\PY{p}{)}
\PY{n}{f2\PYZus{}values} \PY{o}{=} \PY{n}{np}\PY{o}{.}\PY{n}{array}\PY{p}{(}\PY{p}{[}\PY{n}{der\PYZus{}formula2}\PY{p}{(}\PY{n}{f}\PY{p}{,} \PY{n}{x0}\PY{p}{,} \PY{n}{c}\PY{p}{)} \PY{k}{for} \PY{n}{c} \PY{o+ow}{in}  \PY{n}{h}\PY{p}{]}\PY{p}{)}
\PY{n}{columns} \PY{o}{=} \PY{p}{\PYZob{}}
    \PY{l+s+s2}{\PYZdq{}}\PY{l+s+s2}{h}\PY{l+s+s2}{\PYZdq{}} \PY{p}{:} \PY{n}{h}\PY{p}{,}
    \PY{l+s+s2}{\PYZdq{}}\PY{l+s+s2}{real}\PY{l+s+s2}{\PYZdq{}}\PY{p}{:} \PY{n}{ideal}\PY{p}{,}
    \PY{l+s+s2}{\PYZdq{}}\PY{l+s+s2}{f1(2)}\PY{l+s+s2}{\PYZdq{}}\PY{p}{:} \PY{n}{f1\PYZus{}values}\PY{p}{,}
    \PY{l+s+s2}{\PYZdq{}}\PY{l+s+s2}{f2(2)}\PY{l+s+s2}{\PYZdq{}}\PY{p}{:} \PY{n}{f2\PYZus{}values}\PY{p}{,}
    \PY{l+s+s2}{\PYZdq{}}\PY{l+s+s2}{O(h)}\PY{l+s+s2}{\PYZdq{}}\PY{p}{:} \PY{n}{Oh}\PY{p}{,}
    \PY{l+s+s2}{\PYZdq{}}\PY{l+s+s2}{diff\PYZus{}1}\PY{l+s+s2}{\PYZdq{}}\PY{p}{:} \PY{n+nb}{abs}\PY{p}{(}\PY{n}{f1\PYZus{}values} \PY{o}{\PYZhy{}} \PY{n}{ideal}\PY{p}{)}\PY{p}{,}
    \PY{l+s+s2}{\PYZdq{}}\PY{l+s+s2}{O(h\PYZca{}2)}\PY{l+s+s2}{\PYZdq{}}\PY{p}{:} \PY{n}{Oh2}\PY{p}{,}
    \PY{l+s+s2}{\PYZdq{}}\PY{l+s+s2}{diff\PYZus{}2}\PY{l+s+s2}{\PYZdq{}}\PY{p}{:} \PY{n+nb}{abs}\PY{p}{(}\PY{n}{f2\PYZus{}values} \PY{o}{\PYZhy{}} \PY{n}{ideal}\PY{p}{)}
\PY{p}{\PYZcb{}}
\PY{n}{df} \PY{o}{=} \PY{n}{pd}\PY{o}{.}\PY{n}{DataFrame}\PY{p}{(}\PY{n}{columns}\PY{p}{,} \PY{n}{columns}\PY{o}{=}\PY{n}{columns}\PY{o}{.}\PY{n}{keys}\PY{p}{(}\PY{p}{)}\PY{p}{)}
\PY{n}{df}
\end{Verbatim}
\end{tcolorbox}

            \begin{tcolorbox}[breakable, size=fbox, boxrule=.5pt, pad at break*=1mm, opacityfill=0]
\prompt{Out}{outcolor}{3}{\boxspacing}
\begin{Verbatim}[commandchars=\\\{\}]
        h      real      f1(2)     f2(2)       O(h)   f1\_diff    O(h\^{}2)  \textbackslash{}
0  1.0000  7.389056  12.696481  8.683628  10.042768  5.307425  3.347589
1  0.5000  7.389056   9.586876  7.700805   3.045623  2.197820  0.507604
2  0.2500  7.389056   8.394719  7.466266   1.185967  1.005663  0.098831
3  0.1250  7.389056   7.870731  7.408313   0.523306  0.481675  0.021804
4  0.0625  7.389056   7.624851  7.393868   0.245800  0.235795  0.005121

    f2\_diff
0  1.294571
1  0.311749
2  0.077210
3  0.019257
4  0.004812
\end{Verbatim}
\end{tcolorbox}
        
    \(O(h)\) и \(O(h^2)\) - теоретические оценки погрешонсти.
\emph{f1\_diff} и \emph{f2\_diff} - экспериментальные.

Погрешность первого порядка уменьшается примерно во столько раз, во
сколько раз уменьшается \(h\). Погрешность второго порядка - в квадрат
этой величины.

    \hypertarget{ux437ux430ux434ux430ux43dux438ux435-3}{%
\section{Задание 3}\label{ux437ux430ux434ux430ux43dux438ux435-3}}

Неустойчивость численного дифференцирования. Выберите некторую функцию
(например, \(sin (x)\), \(cos (x)\), \(exp (x)\), \(sh (x)\),
\(ln (x)\), \ldots) и некоторую точку \(x\) из области определения
функции. Попробуйте применить формулу
\(y'(x) \approx \frac{y(x+h)−y(x)}{h}\) для стремящейся к нулю
последовательности
\(h = \frac{1}{2}, \frac{1}{4}, \frac{1}{8}, \frac{1}{16}, ...\)). Будет
ли погрешность
\(\varepsilon = \lvert y'(x) − \frac{y(x+h)−y(x)}{h} \rvert\) монотонно
убывать при уменьшении \(h\)? Сравните практический и теоретический
результаты.

    \begin{tcolorbox}[breakable, size=fbox, boxrule=1pt, pad at break*=1mm,colback=cellbackground, colframe=cellborder]
\prompt{In}{incolor}{4}{\boxspacing}
\begin{Verbatim}[commandchars=\\\{\}]
\PY{k}{def} \PY{n+nf}{der\PYZus{}formula1}\PY{p}{(}\PY{n}{f}\PY{p}{,} \PY{n}{x0}\PY{p}{,} \PY{n}{h}\PY{p}{)}\PY{p}{:}
    \PY{k}{return} \PY{p}{(}\PY{n}{f}\PY{p}{(}\PY{n}{x0}\PY{o}{+}\PY{n}{h}\PY{p}{)} \PY{o}{\PYZhy{}} \PY{n}{f}\PY{p}{(}\PY{n}{x0}\PY{p}{)}\PY{p}{)} \PY{o}{/} \PY{n}{h}

\PY{k}{def} \PY{n+nf}{Rh}\PY{p}{(}\PY{n}{F2}\PY{p}{,} \PY{n}{x0}\PY{p}{,} \PY{n}{h}\PY{p}{)}\PY{p}{:}
    \PY{n}{x} \PY{o}{=} \PY{n}{np}\PY{o}{.}\PY{n}{linspace}\PY{p}{(}\PY{n}{x0}\PY{p}{,} \PY{n}{x0}\PY{o}{+}\PY{n}{h}\PY{p}{)}
    \PY{n}{y} \PY{o}{=} \PY{n+nb}{abs}\PY{p}{(}\PY{n}{F2}\PY{p}{(}\PY{n}{x}\PY{p}{)} \PY{o}{*} \PY{n}{h} \PY{o}{/} \PY{l+m+mi}{2}\PY{p}{)}
    \PY{k}{return} \PY{n+nb}{max}\PY{p}{(}\PY{n}{y}\PY{p}{)}

\PY{n}{f} \PY{o}{=} \PY{n}{np}\PY{o}{.}\PY{n}{exp}
\PY{n}{F2} \PY{o}{=} \PY{n}{np}\PY{o}{.}\PY{n}{exp}
\PY{n}{MAX\PYZus{}ITER} \PY{o}{=} \PY{l+m+mi}{500}

\PY{n}{h} \PY{o}{=} \PY{p}{[}\PY{l+m+mi}{1}\PY{o}{/}\PY{l+m+mi}{2}\PY{p}{]}
\PY{n}{x0} \PY{o}{=} \PY{l+m+mi}{2}
\PY{n}{theor\PYZus{}err} \PY{o}{=} \PY{p}{[}\PY{p}{]}
\PY{n}{exper\PYZus{}err} \PY{o}{=} \PY{p}{[}\PY{p}{]}
\PY{n}{df} \PY{o}{=} \PY{p}{[}\PY{p}{]}
\PY{k}{for} \PY{n}{i} \PY{o+ow}{in} \PY{n+nb}{range}\PY{p}{(}\PY{n}{MAX\PYZus{}ITER}\PY{p}{)}\PY{p}{:}
    \PY{n}{theor\PYZus{}err}\PY{o}{.}\PY{n}{append}\PY{p}{(}\PY{n}{Rh}\PY{p}{(}\PY{n}{F2}\PY{p}{,} \PY{n}{x0}\PY{p}{,} \PY{n}{h}\PY{p}{[}\PY{o}{\PYZhy{}}\PY{l+m+mi}{1}\PY{p}{]}\PY{p}{)}\PY{p}{)}
    \PY{n}{df}\PY{o}{.}\PY{n}{append}\PY{p}{(}\PY{n}{der\PYZus{}formula1}\PY{p}{(}\PY{n}{f}\PY{p}{,} \PY{n}{x0}\PY{p}{,} \PY{n}{h}\PY{p}{[}\PY{o}{\PYZhy{}}\PY{l+m+mi}{1}\PY{p}{]}\PY{p}{)}\PY{p}{)}
    \PY{n}{exper\PYZus{}err}\PY{o}{.}\PY{n}{append}\PY{p}{(}\PY{n+nb}{abs}\PY{p}{(}\PY{n}{df}\PY{p}{[}\PY{o}{\PYZhy{}}\PY{l+m+mi}{1}\PY{p}{]}\PY{o}{\PYZhy{}}\PY{n}{f}\PY{p}{(}\PY{n}{x0}\PY{p}{)}\PY{p}{)}\PY{p}{)}
    \PY{k}{if} \PY{n}{df}\PY{p}{[}\PY{o}{\PYZhy{}}\PY{l+m+mi}{1}\PY{p}{]} \PY{o}{==} \PY{l+m+mi}{0}\PY{p}{:}
        \PY{k}{break}
    \PY{n}{h}\PY{o}{.}\PY{n}{append}\PY{p}{(}\PY{n}{h}\PY{p}{[}\PY{o}{\PYZhy{}}\PY{l+m+mi}{1}\PY{p}{]}\PY{o}{/}\PY{l+m+mi}{2}\PY{p}{)}

\PY{n}{h} \PY{o}{=} \PY{n}{np}\PY{o}{.}\PY{n}{array}\PY{p}{(}\PY{n}{h}\PY{p}{)}
\PY{n}{theor\PYZus{}err} \PY{o}{=} \PY{n}{np}\PY{o}{.}\PY{n}{array}\PY{p}{(}\PY{n}{theor\PYZus{}err}\PY{p}{)}
\PY{n}{exper\PYZus{}err} \PY{o}{=} \PY{n}{np}\PY{o}{.}\PY{n}{array}\PY{p}{(}\PY{n}{exper\PYZus{}err}\PY{p}{)}
\PY{n}{df} \PY{o}{=} \PY{n}{np}\PY{o}{.}\PY{n}{array}\PY{p}{(}\PY{n}{df}\PY{p}{)}

\PY{n}{columns} \PY{o}{=} \PY{p}{\PYZob{}}
    \PY{l+s+s2}{\PYZdq{}}\PY{l+s+s2}{h}\PY{l+s+s2}{\PYZdq{}}\PY{p}{:} \PY{n}{h}\PY{p}{,}
    \PY{l+s+s2}{\PYZdq{}}\PY{l+s+s2}{df}\PY{l+s+s2}{\PYZdq{}}\PY{p}{:} \PY{n}{df}\PY{p}{,}
    \PY{l+s+s2}{\PYZdq{}}\PY{l+s+s2}{theor\PYZus{}err}\PY{l+s+s2}{\PYZdq{}}\PY{p}{:} \PY{n}{theor\PYZus{}err}\PY{p}{,}
    \PY{l+s+s2}{\PYZdq{}}\PY{l+s+s2}{exper\PYZus{}err}\PY{l+s+s2}{\PYZdq{}}\PY{p}{:} \PY{n}{exper\PYZus{}err}
\PY{p}{\PYZcb{}}
\PY{n}{dframe} \PY{o}{=} \PY{n}{pd}\PY{o}{.}\PY{n}{DataFrame}\PY{p}{(}\PY{n}{columns}\PY{p}{,} \PY{n}{columns}\PY{o}{=}\PY{n}{columns}\PY{o}{.}\PY{n}{keys}\PY{p}{(}\PY{p}{)}\PY{p}{)}
\PY{n}{dframe}
\end{Verbatim}
\end{tcolorbox}

            \begin{tcolorbox}[breakable, size=fbox, boxrule=.5pt, pad at break*=1mm, opacityfill=0]
\prompt{Out}{outcolor}{4}{\boxspacing}
\begin{Verbatim}[commandchars=\\\{\}]
               h        df     theor\_err     exper\_err
0   5.000000e-01  9.586876  3.045623e+00  2.197820e+00
1   2.500000e-01  8.394719  1.185967e+00  1.005663e+00
2   1.250000e-01  7.870731  5.233061e-01  4.816750e-01
3   6.250000e-02  7.624851  2.458003e-01  2.357947e-01
4   3.125000e-02  7.505722  1.191189e-01  1.166661e-01
5   1.562500e-02  7.447085  5.863607e-02  5.802884e-02
6   7.812500e-03  7.417995  2.908988e-02  2.893881e-02
7   3.906250e-03  7.403507  1.448823e-02  1.445056e-02
8   1.953125e-03  7.396277  7.229982e-03  7.220575e-03
9   9.765625e-04  7.392665  3.611463e-03  3.609112e-03
10  4.882812e-04  7.390860  1.804850e-03  1.804262e-03
11  2.441406e-04  7.389958  9.022046e-04  9.020578e-04
12  1.220703e-04  7.389507  4.510472e-04  4.510105e-04
13  6.103516e-05  7.389282  2.255099e-04  2.255007e-04
14  3.051758e-05  7.389169  1.127515e-04  1.127492e-04
15  1.525879e-05  7.389112  5.637488e-05  5.637430e-05
16  7.629395e-06  7.389084  2.818723e-05  2.818701e-05
17  3.814697e-06  7.389070  1.409356e-05  1.409342e-05
18  1.907349e-06  7.389063  7.046766e-06  7.046569e-06
19  9.536743e-07  7.389060  3.523380e-06  3.522910e-06
20  4.768372e-07  7.389058  1.761689e-06  1.760848e-06
21  2.384186e-07  7.389057  8.808443e-07  8.816792e-07
22  1.192093e-07  7.389057  4.404221e-07  4.420950e-07
23  5.960464e-08  7.389056  2.202110e-07  2.111270e-07
24  2.980232e-08  7.389056  1.101055e-07  1.068189e-07
25  1.490116e-08  7.389056  5.505276e-08  4.721422e-08
26  7.450581e-09  7.389056  2.752638e-08  1.239043e-08
27  3.725290e-09  7.389056  1.376319e-08  1.315997e-07
28  1.862645e-09  7.389056  6.881595e-09  1.068189e-07
29  9.313226e-10  7.389056  3.440797e-09  1.068189e-07
30  4.656613e-10  7.389055  1.720399e-09  8.468555e-07
31  2.328306e-10  7.389057  8.601993e-10  1.060493e-06
32  1.164153e-10  7.389053  4.300997e-10  2.754204e-06
33  5.820766e-11  7.389053  2.150498e-10  2.754204e-06
34  2.910383e-11  7.389038  1.075249e-10  1.801299e-05
35  1.455192e-11  7.389038  5.376246e-11  1.801299e-05
36  7.275958e-12  7.389038  2.688123e-11  1.801299e-05
37  3.637979e-12  7.388916  1.344061e-11  1.400833e-04
38  1.818989e-12  7.389160  6.720307e-12  1.040573e-04
39  9.094947e-13  7.388672  3.360154e-12  3.842239e-04
40  4.547474e-13  7.388672  1.680077e-12  3.842239e-04
41  2.273737e-13  7.386719  8.400384e-13  2.337349e-03
42  1.136868e-13  7.390625  4.200192e-13  1.568901e-03
43  5.684342e-14  7.390625  2.100096e-13  1.568901e-03
44  2.842171e-14  7.375000  1.050048e-13  1.405610e-02
45  1.421085e-14  7.375000  5.250240e-14  1.405610e-02
46  7.105427e-15  7.375000  2.625120e-14  1.405610e-02
47  3.552714e-15  7.250000  1.312560e-14  1.390561e-01
48  1.776357e-15  7.500000  6.562800e-15  1.109439e-01
49  8.881784e-16  7.000000  3.281400e-15  3.890561e-01
50  4.440892e-16  6.000000  1.640700e-15  1.389056e+00
51  2.220446e-16  0.000000  8.203500e-16  7.389056e+00
\end{Verbatim}
\end{tcolorbox}
        
    При малых \(h\) (порядка \(10^{-16}\)) ЭВМ считает \(x + h = x\).
Следовательно, численная производная в таких случаях будет равна \(0\).
В таком случае погрешность будет максимальной и будет равна \(y'(x_0)\).
До тех пор, пока это не произойдёт погрешность действительно будет
монотонно убывать.


    % Add a bibliography block to the postdoc
    
    
    
\end{document}
